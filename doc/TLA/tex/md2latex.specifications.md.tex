\documentclass[a4paper, 11pt, titlepage, openany]{article}
\usepackage{tlatex} 
%: Import packages from src/md2latex_doc.packages.tex. 
% Packages: BEGINNING 

\usepackage{cite}
\usepackage{amsmath, 
%physics, 
%amssymb, 
%mathrsfs, 
%amsthm, 
mathspec, 
%fouridx, 
% stix
}
\usepackage{tlatex}
\usepackage[xetex]{hyperref}
\hypersetup{
    pdfborder = {0 0 0},
    colorlinks=true, %set true if you want colored links
    linktoc=all,%set to all if you want both sections and subsections linked
    linkcolor=blue,%choose some color if you want links to stand out
}
\usepackage[bottom=3cm,top=2cm]{geometry}
%\usepackage{array}
\usepackage{xltabular,colortbl}


% FOR XELATEX. SKIP IF ERROR MESSAGE
\usepackage{xltxtra,xunicode, xcolor}
% for HEVEA- Uncomment before using XeTeX
%\usepackage[utf8]{inputenc}
% ligntskyblue #87cefa
% light steel blue b0c4de
% pale aqua bcd4e6
% ALice blue f0f8ff
% Ghost white f8f8ff
% White smoke f5f5f5
\definecolor{boxshade}{HTML}{f8f8ff}
%\usepackage[xetex]{xcolor}   %May be necessary if you want to color links



\usepackage{lastpage}
\usepackage{tocloft}
%\usepackage{array}
\usepackage{framed}
%\usepackage[left=4cm,top=10cm,right=3cm,bottom=3cm]{geometry}
\usepackage{graphicx}
\usepackage{fancyhdr, tikz, tikzpagenodes}
%\usepackage{amsmath,amssymb,mathrsfs,amsthm, mathspec}
\usepackage{polyglossia}
\usepackage{sectsty}
\usepackage{listings}
\usepackage[labelsep=quad]{caption}
\usepackage{float}

\floatstyle{boxed} 
\restylefloat{figure}

% Packages: END


%: Fonts.
\defaultfontfeatures{Mapping=tex-text}
\newfontfamily{\fw}{CMU Typewriter Text Light}
\def\fwfont{CMU Typewriter Text Light}
\def\mainfont{CMU Serif}
\setmainfont{\mainfont}
\newfontfamily\mainfontLARGE[SizeFeatures={Size=36}]{\mainfont}
\newfontfamily\mainfontLarge[SizeFeatures={Size=24}]{\mainfont}
\setmathfont(Latin)[Uppercase=Regular,Lowercase=Regular]{\mainfont}
\setmathfont(Greek)[Uppercase=Regular,Lowercase=Regular]{\mainfont}
\setmathrm{\mainfont}
\setmathbb{\mainfont}

%: Default language and dependent settings.
\setdefaultlanguage[]{english}
\newcommand{\currentdate}{ \today}
\def\pagenumbering{ of }
\def\nametableofcontents{Contents}

%: Setting PATH.
% PATHROOT
\def\PATHROOT{../src/}
% PATHIMAGES
\def\PATHIMAGES{../src/img/}

%: Custom (color, font,…) for section, subsection,….
\sectionfont{\color{blue}}
\renewcommand{\thesection}{\arabic{section}}
\renewcommand{\thesubsection}{\thesection.\alph{subsection}}

\title{md2LaTeX specifications in TLA+}
\author{Git Cordier \\\texttt{admin@gcordier.eu}}\setlength\parindent{0pt}

\begin{document}
%tlatex: rennewommads
%\renewcommand{\@w}[1]{#1} 

\maketitle
%: Table of comments (aka "toc").
\tableofcontents\newpage
\section{Overview}
md2LaTex (this "forked" version) needs two inputs: the {\fw .md} file, 
of course, and to preferences file (in practice, a JSON) that encodes the user
choices (titlepage, table of contents, language, graphic design,…).
This document specifies format and parsing process of such a preferences file. 
\section{Around the specifications}
NAME stands for the project's name. The naming convention (but it is off specs) is                      
\begin{itemize}
  \item[-]{NAME.md.pdf: The pdf output, md stands for \textit{main document} - 
    think of it as a main function in a C programm. }
  \item[-]{NAME.md.md:The markdown main document (recursive imports are allowed.}
  \item[-]{NAME. preferences.json: the preferences file}
\end{itemize}
and so on. This convention is followed in the {\fw .tla} documents.
\section{The specifications}
Following the \href{https://lamport.azurewebsites.net/tla/formal-methods-amazon.pdf}{Use of Formal Methods at Amazon Web Services}, 
\begin{quote}
    \textit{In TLA+, correctness properties and system designs are just steps on a ladder of abstraction, 
    with correctness properties occupying higher levels,
    systems designs and algorithms in the middle, 
    and executable code and hardware at the lower levels.}
\end{quote}
I always beared that in mind as I was writing the specs and you may read them as if your were standing on this ladder: 
Starting from the highest bar and stepping down to the lowest. 
\input{md2latexSpecificationsAll.tex}
\newpage
\section{Aknowledgments} 
\begin{itemize}
  \item[-]{\href{https://github.com/lepture/mistune}{Lepture}}
  \item[-]{\href{https://github.com/kavinyao/md2latex}{Kavin Yao}}
  \item[-]{The TLA typesetting was performed with the{\fw tlatex} \LaTeX package, written by Leslie Lamport;
          Documented by the author in 
          \href{https://lamport.azurewebsites.net/tla/book.html}{\textit{Specifying Systems}}.
  }
\end{itemize}
\end{document}
