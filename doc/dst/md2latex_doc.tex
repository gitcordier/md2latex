\documentclass[a4paper, 11pt, titlepage, openany]{article}


%: Import packages from src/md2latex_doc.packages.tex. 
% Packages: BEGINNING 


\usepackage[xetex]{xcolor}   %May be necessary if you want to color links
\usepackage{hyperref}
\hypersetup{
    pdfborder = {0 0 0},
    colorlinks=true, %set true if you want colored links
    linktoc=all,     %set to all if you want both sections and subsections linked
    linkcolor=corporate_blue,  %choose some color if you want links to stand out
}

% FOR XELATEX. SKIP IF ERROR MESSAGE
\usepackage{xltxtra,xunicode}
% for HEVEA- Uncomment before using XeTeX
%\usepackage[utf8]{inputenc}

\usepackage{lastpage}
\usepackage{tocloft}

%\usepackage{array}
%\usepackage{xltabular,colortbl}
\usepackage{framed}
\usepackage[bottom=3cm,top=2cm]{geometry}
%\usepackage[left=4cm,top=10cm,right=3cm,bottom=3cm]{geometry}
\usepackage{graphicx}
\usepackage{fancyhdr, tikz, tikzpagenodes}
%\usepackage{amsmath,amssymb,mathrsfs,amsthm, mathspec}
\usepackage{polyglossia}

%\usepackage{xcolor}
\usepackage{sectsty}
%\usepackage{colortbl}
%\usepackage{hyperref}

\usepackage{cite}
\usepackage{listings}

\usepackage[labelsep=quad]{caption}
\usepackage{float}
\floatstyle{boxed} 
\restylefloat{figure}

% Packages: END


%: Fonts.
\defaultfontfeatures{Mapping=tex-text}
\newfontfamily{\fw}{CMU Typewriter Text Light}
\def\mainfont{Arial}
\setmainfont{\mainfont}
\newfontfamily\mainfontLARGE[SizeFeatures={Size=36}]{\mainfont}
\newfontfamily\mainfontLarge[SizeFeatures={Size=24}]{\mainfont}

%: Define colors.
\definecolor{corporate_blue}{HTML}{66ACC9}
\definecolor{corporate_grey}{HTML}{f5f5f5}



%: Default language and dependent settings.
\setdefaultlanguage[]{english}
\newcommand{\currentdate}{ \today}
\def\pagenumbering{ of }
\def\nametableofcontents{Contents}

%: Setting PATH.
% PATHROOT
\def\PATHROOT{../src/}
% PATHIMAGES
\def\PATHIMAGES{../src/img/}

%: Custom (color, font,…) for section, subsection,….
\sectionfont{\color{corporate_blue}}
\renewcommand{\thesection}{\arabic{section}}
\renewcommand{\thesubsection}{\thesection.\alph{subsection}}

%: Pimp my page: package fancy.
% FANCY: BEGINNING 
\pagestyle{fancy}
\def\footimg{\PATHROOT img/unicorn.jpg}
\fancyhf{}
\fancyfoot[LE,LO]{
\vspace{0.1cm}
  \includegraphics[scale=0.1]{\footimg}
}
\fancyfoot[RE,RO]{
  \thepage \pagenumbering \color{black}{\pageref{LastPage}}
}

\renewcommand{\headrulewidth}{0pt}
\fancyhead[LO,LE]{\color{gray}{\currenttitle}}
\fancyhead[RO,RE]{\color{gray}{\today}}
%
\renewcommand{\caption}[1]{\center{
    \textbf{Table }\addtocounter{table}{1}\thetable. #1
  }
}
% FANCY: END


\newenvironment{blockquote}{%
  \par%
  \medskip
  \leftskip=4em\rightskip=2em%
  \noindent\ignorespaces}{%
  \par\medskip}
\def\currenttitle{Markdown to {{\color{corporate_blue}{\LaTeX{} or \XeLaTeX{}}}} Test Document}\title{\currenttitle}

\author{Git Cordier \\\texttt{admin@gcordier.eu}}\setlength\parindent{0pt}

\begin{document}

% TITLE PAGE: BEGINNING 
\def\coverimg{\PATHIMAGES unicorn.jpg}
\usetikzlibrary{calc}
%\renewcommand{\headrulewidth}{0pt}
\def\cover{
  \begin{tikzpicture}[remember picture,overlay]
    \draw  let \p1=($(current page.north)-(current page header area.south)$),
    \n1={veclen(\x1,\y1)} in
      node [inner sep=128,outer sep=0, inner ysep=120, below right] 
      at (current page.north west){
        \includegraphics[width=0.5\paperwidth]{\coverimg}
      };
  \end{tikzpicture}
}

\begin{titlepage}
	
	%\includegraphics[width=\paperwidth]{\coverimg}\par\vspace{1cm}
  \begin{minipage}[c]{\paperwidth}
    \cover
    \vspace{16cm}
  \end{minipage}
  \begin{minipage}[r]{\textwidth}
    \begin{flushright}
    {\scshape\mainfontLARGE \color{corporate_blue}{\currenttitle}\par}
    \vspace{1 cm}
    {\scshape\mainfontLarge \color{corporate_blue}{Documentation}\par}
    \vspace{2 cm}
    {\scshape\mainfontLarge \color{gray}{ The Unicorn Company} \par}
    {\scshape\mainfontLarge \color{gray}{ \currentdate } \par}
    \end{flushright}
    
  \end{minipage}
\end{titlepage}
% TITLE PAGE: END 


%: Foreword.
% FOREWORD: BEGINNING
\section*{Forewords}
This is the corporate version of the md2\LaTeX full manual.
It is intended to show that you can end up with all kinds of graphic design, 
as you are writing, only, a .md file. 
Your manager will keep sure that you are good at MSWord.


\vspace{7cm}
{\color{corporate_blue}{THE UNICORN COMPANY}}\\
\href{https://github.com/gitcordier}{\underline{http://theunicorncompany.somedomain}}\\
\href{mailto:admin@gcordier.eu}{\color{gray}{\underline{admin@gcordier.eu}}}\\
\\
{\fw Copyright (c) <2020> <gitcordier>

Permission is hereby granted, free of charge, to any person obtaining a copy
of this software and associated documentation files (the "Software"), to deal
in the Software without restriction, including without limitation the rights
to use, copy, modify, merge, publish, distribute, sublicense, and/or sell
copies of the Software, and to permit persons to whom the Software is
furnished to do so, subject to the following conditions:

The above copyright notice and this permission notice shall be included in all
copies or substantial portions of the Software.

THE SOFTWARE IS PROVIDED "AS IS", WITHOUT WARRANTY OF ANY KIND, EXPRESS OR
IMPLIED, INCLUDING BUT NOT LIMITED TO THE WARRANTIES OF MERCHANTABILITY,
FITNESS FOR A PARTICULAR PURPOSE AND NONINFRINGEMENT. IN NO EVENT SHALL THE
AUTHORS OR COPYRIGHT HOLDERS BE LIABLE FOR ANY CLAIM, DAMAGES OR OTHER
LIABILITY, WHETHER IN AN ACTION OF CONTRACT, TORT OR OTHERWISE, ARISING FROM,
OUT OF OR IN CONNECTION WITH THE SOFTWARE OR THE USE OR OTHER DEALINGS IN THE
SOFTWARE.}
%FOREWORD: END
\newpage


%: Table of comments (aka "toc").
\tableofcontents\newpage



%  Some LaTeX commands . TODO: md2latex.py 
% \newcommand{\words}[1]{#1$^\ast$} 

%  Content: BEGINNING 

\newcommand{\var}[1]{\${#1}}

\section{Introduction: Why I do this, by Kavin Yao}

I love \LaTeX{} for its pretty typesetting, but not like its verbose syntax very much. 95\% of the time, I only use a very small subset of \LaTeX{} and really miss the simplicity of markdown every time I have to type in plain \LaTeX{}.

I also use \href{http://www.texmacs.org/tmweb/home/welcome.en.html}{TeX{}macs}. Its a great tool and I love it. However, the source code of TeX{}macs documents, with an XML-like structure, is not human-readable. It's not good for source control, either.

So, my conclusion is that, since what I mostly use in \LaTeX{} can be mapped to markdown, why not write document in markdown and convert it to \LaTeX{}? I find \href{http://johnmacfarlane.net/pandoc/}{Pandoc} but it's too cryptic to use[note1].

And an idea bubbles up in my head: why not write my own converter from markdown to \LaTeX{}?

I have a great start point: \href{https://github.com/lepture/mistune}{mistune}. It's a fast, clean implementation of markdown with a killer feature - footnote. I tend to use footnote much in \LaTeX{}.

\subsection{Plan}

My current plan of the converter includes:

\begin{itemize}
    \item[-] 

title and author (with meta header)
    \item[-] 

sections (headers in markdown)
    \item[-] 

lists (ordered with \texttt{enumerate} and unordered with \texttt{itemize} package)
    \item[-] 

emphasize, strong and monospace styles
    \item[-] 

hyperlink
    \item[-] 

footnote
    \item[-] 

math\footnote{since there's no inline math in markdown, it is translated as is.}
\end{itemize}

Now let's see how much we can do...

Test quote:

\begin{blockquote}

Steve Jobs: stay foolish, stay hungry.
\end{blockquote}

Test \texttt{code}:

\begin{verbatim}

int main()

{

    printf("Hello world!");

}
\end{verbatim}

%  Content: END 

\noindent\rule{\textwidth}{0.4pt}\end{document}
