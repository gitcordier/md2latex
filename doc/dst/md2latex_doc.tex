\documentclass[a4paper, 11pt, titlepage, openany]{article}


%: Import packages from src/md2latex_doc.packages.tex. 
% Packages: BEGINNING 

\usepackage{cite}
\usepackage{amsmath, 
physics, 
amssymb, 
mathrsfs, 
amsthm, 
mathspec, 
fouridx, 
% stix
}

\usepackage[xetex]{hyperref}
\hypersetup{
    pdfborder = {0 0 0},
    colorlinks=true, %set true if you want colored links
    linktoc=all,%set to all if you want both sections and subsections linked
    linkcolor=blue,%choose some color if you want links to stand out
}
\usepackage[bottom=3cm,top=2cm]{geometry}
%\usepackage{array}
\usepackage{xltabular,colortbl}


% FOR XELATEX. SKIP IF ERROR MESSAGE
\usepackage{xltxtra,xunicode, xcolor}
% for HEVEA- Uncomment before using XeTeX
%\usepackage[utf8]{inputenc}

%\usepackage[xetex]{xcolor}   %May be necessary if you want to color links



\usepackage{lastpage}
\usepackage{tocloft}
%\usepackage{array}
\usepackage{framed}
%\usepackage[left=4cm,top=10cm,right=3cm,bottom=3cm]{geometry}
\usepackage{graphicx}
\usepackage{fancyhdr, tikz, tikzpagenodes}
%\usepackage{amsmath,amssymb,mathrsfs,amsthm, mathspec}
\usepackage{polyglossia}
\usepackage{sectsty}
\usepackage{listings}
\usepackage[labelsep=quad]{caption}
\usepackage{float}

\floatstyle{boxed} 
\restylefloat{figure}

% Packages: END


%: Fonts.
\defaultfontfeatures{Mapping=tex-text}
\newfontfamily{\fw}{CMU Typewriter Text Light}
\def\fwfont{CMU Typewriter Text Light}
\def\mainfont{CMU Serif}
\setmainfont{\mainfont}
\newfontfamily\mainfontLARGE[SizeFeatures={Size=36}]{\mainfont}
\newfontfamily\mainfontLarge[SizeFeatures={Size=24}]{\mainfont}
\setmathfont(Latin)[Uppercase=Regular,Lowercase=Regular]{\mainfont}
\setmathfont(Greek)[Uppercase=Regular,Lowercase=Regular]{\mainfont}
\setmathrm{\mainfont}
\setmathbb{\mainfont}

%: Default language and dependent settings.
\setdefaultlanguage[]{english}
\newcommand{\currentdate}{ \today}
\def\pagenumbering{ of }
\def\nametableofcontents{Contents}

%: Setting PATH.
% PATHROOT
\def\PATHROOT{../src/}
% PATHIMAGES
\def\PATHIMAGES{../src/img/}

%: Custom (color, font,…) for section, subsection,….
\sectionfont{\color{blue}}
\renewcommand{\thesection}{\arabic{section}}
\renewcommand{\thesubsection}{\thesection.\alph{subsection}}


\newenvironment{blockquote}{%
  \par%
  \medskip
  \leftskip=4em\rightskip=2em%
  \noindent\ignorespaces}{%
  \par\medskip}
\def\currenttitle{Markdown to \LaTeX or \XeLaTeX Test Document}\title{\currenttitle}

\author{Git Cordier \\\texttt{admin@gcordier.eu}}\setlength\parindent{0pt}

\begin{document}
\maketitle
%: Table of comments (aka "toc").
\tableofcontents\newpage



%  Some LaTeX commands . TODO: md2latex.py 
% \renewcommand{\words}[1]{#1$^\ast$} 

%  Content: BEGINNING 

\newcommand{\vardoc}[1]{\$\{#1\}}

\section{Introduction: Why I do this, by Kavin Yao}

I love \LaTeX{} for its pretty typesetting, but not like its verbose syntax very much. 95\% of the time, I only use a very small subset of \LaTeX{} and really miss the simplicity of markdown every time I have to type in plain \LaTeX{}.

I also use \href{http://www.texmacs.org/tmweb/home/welcome.en.html}{TeX{}macs}. Its a great tool and I love it. However, the source code of TeX{}macs documents, with an XML-like structure, is not human-readable. It's not good for source control, either.

So, my conclusion is that, since what I mostly use in \LaTeX{} can be mapped to markdown, why not write document in markdown and convert it to \LaTeX{}? I find \href{http://johnmacfarlane.net/pandoc/}{Pandoc} but it's too cryptic to use\footnote{Pandoc is too omnipotent and lack the simplicity I prefer.}.

And an idea bubbles up in my head: why not write my own converter from markdown to \LaTeX{}?

I have a great start point: \href{https://github.com/lepture/mistune}{mistune}. It's a fast, clean implementation of markdown with a killer feature - footnote. I tend to use footnote much in \LaTeX{}.

\subsection{Plan}

My current plan of the converter includes:

\begin{itemize}
    \item[-] 

title and author (with meta header)
    \item[-] 

sections (headers in markdown)
    \item[-] 

lists (ordered with \texttt{enumerate} and unordered with \texttt{itemize} package)
    \item[-] 

emphasize, strong and monospace styles
    \item[-] 

hyperlink
    \item[-] 

footnote
    \item[-] 

math\footnote{since there's no inline math in markdown, it is translated as is.}\end{itemize}

Now let's see how much we can do...

Test quote:

\begin{blockquote}

Steve Jobs: stay foolish, stay hungry.\end{blockquote}

Test \texttt{code}:


int main()

{

    printf("Hello world!");

}

%  Content: END 

\section{Why I do this}

\section{What is mapped (and what is not)}

what is mapped LateX code (math) case 

The current line was typeset from \LaTeX {\fw commands}, 

including AMS ones: $12^3 + 1^3 = 1729 = 9^3 + 10^3$. You can also use environments, e.g.

\begin{align}e^{i\pi} +1 = 0\end{align}

TODO 

summary as a table

%\setcounter{section}{0}

\begin{table}[htp]

  \begin{center}

\begin{tabular}{|l|l|l|l|}\hline\rowcolor{white}

  %
  \begin{tabular}{l}

    \color{blue}{input}

  \end{tabular} 

   &

  \begin{tabular}{l}

    \color{blue}{output}

  \end{tabular} 

   & 

  \begin{tabular}{l}

    \color{blue}{in 0.0.1}

  \end{tabular} 

  &

  \begin{tabular}{l}

    \color{blue}{in 1.0.0}

  \end{tabular}

  \\\hline

  %
  regular text        & regular text      & yes & yes \\
  title and author    & title and author  & yes & yes \\
  titlepage           & titlepage         & no  & yes \\
  sections%
                      & yes                & yes          & yes \\
  lists %
                      & yes                & yes         & yes \\
  emphasize, strong   & yes                & yes         & yes \\
  hyperlink           & yes                & yes         & yes \\
  footnote            & yes                & yes         & yes \\
  HTML comment        & LaTeX comment     & no          & yes \\
  LaTeX code         & executable LaTeX code & no       & yes \\
  LaTeX code         & quoted LaTeX code  & yes         & yes

   % END\
  \\\hline
\end{tabular}
\end{center}
\end{table}

not mapped = color/ font of a given piece of characters (see title!)

By the very definition of what md is, i don' t see any complete mapping

workaround: defining commands = language extensions, like what we do with inputmd

\section{comment}

note that comment may start with %: . So that TeXShop users can get the document structure

\subsection{XeLaTeX or LaTeX?}

I use \XeLaTeX, which is \XeTeX compiling \LaTeX code.

From the \href{https://en.wikipedia.org/wiki/XeTeX}{{{\XeTeX{}}}} wikipedia page:

\begin{blockquote}

{{\XeTeX{}}} is a {{\TeX{}}} typesetting engine using Unicode and supporting modern font technologies […]. 

It was originally written by Jonathan Kew and is distributed under the X11 free software license.

It natively supports Unicode and the input file is assumed to be in UTF-8 encoding by default. 

\textbf{XeTeX can use any fonts installed in the operating system without configuring TeX font metrics}[…].\end{blockquote}

This is the reason why I use \XeLaTeX {{$\uparrow$}}.

%  Euler's identity 

%  breaking lines brings trouble: TODO

\begin{align}  e^{i\pi} + 1 = 0  \end{align}

%  For now you cant nest \lstset{tabsize = 4,     showstringspaces = false,     commentstyle = \color{gray},     keywordstyle = \color{purple},     stringstyle = \color{red},     rulecolor = \color{black},     basicstyle = \small \fw,     breaklines = true,     numberstyle = \tiny,}{\color{blue}     \begin{lstlisting}[language = TeX,     frame = trBL , firstnumber = last] inside \© markups

\lstset{tabsize = 4,     showstringspaces = false,     commentstyle = \color{gray},     keywordstyle = \color{purple},     stringstyle = \color{red},     rulecolor = \color{black},     basicstyle = \small \fw,     breaklines = true,     numberstyle = \tiny,}{\color{blue}     \begin{lstlisting}[language = TeX,     frame = trBL , firstnumber = last]

% Euler's identity

\begin{align} 

  e^{i\pi} + 1 = 0.
\end{align}
\end{lstlisting}}

\section{General policy}

Either you don't say it, either you say it clearly

it is true if and only if it is explicitely stated that it is true

%\setcounter{section}{0}

\begin{table}[htp]

  \begin{center}

\begin{tabular}{|l|l|l|l|}\hline\rowcolor{white}

  %
  \begin{tabular}{l}

    \color{blue}{object}

  \end{tabular} 

   &

  \begin{tabular}{l}

    \color{blue}{meaning}

  \end{tabular} 

  \\\hline

  %
  True        & True  \\
  "True"      & True  \\
  "true"      & True  \\
  "yes"       & True  \\
  "Yes"       & True  \\
  "YES"       & True  \\
  1           & True  \\
  False       & False \\
  "False"     & False \\
  "true"      & False \\
  "no"        & False \\
  "No"        & False \\
  "No"        & False \\
  0           & False \\
  empty       & False \\
  None        & False \\\end{tabular}
\end{center}
\end{table}

no = false, yes = true

no = $\emptyset$, False, false

\section{naming conventions}

\subsection{paths}

\subsection{keys}

%\setcounter{section}{0}

\newcommand{\words}[1]{{#1}$^\ast$}

\begin{table}[htp]

  \begin{center}

\begin{tabular}{|l|l|l|l|}\hline\rowcolor{white}

  %
  \begin{tabular}{l}

    \color{blue}{key}

  \end{tabular} 

   &

  \begin{tabular}{l}

    \color{blue}{syntax ($\small{\in}$)}

  \end{tabular} 

   & 

  \begin{tabular}{l}

    \color{blue}{meaning}

  \end{tabular} 

  &

  \begin{tabular}{l}

    \color{blue}{optional}

  \end{tabular}

  \\\hline

  %
   {\fw documentclass} & Dict     & document class       & yes \\
   {\fw titlepafe} & $False \cup Path$& document class       & yes \\
   {\fw packages} & \words{ASCII} & path of packages     & {\bf no} \\
   {\fw fonts}    &  Dict         & set the fonts        & yes \\
   {\fw colors}   &  Dict         & set extra colors     & yes \\
   {\fw language} &  Dict    & default language & {\bf no} \\
                  &               & dependent settings.  &     \\
   {\fw fancy}    & \words{ASCII} & page foot seetings.  & yes \\
   {\fw custom}.  & \words{ASCII} & section, subsection  & yes \\
   {\fw foreword} & $Bool^{\{ \text{\fw Y/N} \}} \times Path^{\{ \text{\fw path} \}}$ &&\\
   {\fw toc}      & \words{ASCII} & toc                  & yes \\
   {\fw annex}    & \words{ASCII} & including annex      & yes 

   % END

  \\\hline
\end{tabular}
\end{center}
\end{table}

\subsubsection{Y/N}

\subsection{files}

\subsubsection{md to *tex}

\subsubsection{log}

\subsubsection{script}

\section{The structure of a md2latex}

\begin{align} %
& \text{{\fw md2latex/doc}} \\
& \text{{\fw md2latex/doc/dst}} \\
& \text{{\fw md2latex/doc/dst/\$\{name\}.tex}} \\
& \text{{\fw md2latex/doc/dst/\$\{name\}.pdf}} \\
& \text{{\fw md2latex/doc/src}} \\
&\text{{\fw md2latex/md2latex}} \\
&\text{{\fw md2latex/md2latex}}
\end{align}

\subsection{Root}

\subsubsection{{\fw \vardoc{name}.run.sh}}

{\color{orange}{{@}{\fw optional}}}

\subsubsection{{\fw \vardoc{name}.preferences.json}}

{\color{red}{{@}{\fw !optional}}}

\subsection{{{\fw src}}}

{\color{red}{{@}{\fw !optional}}}

\subsubsection{{{\fw content/}}}

{\color{orange}{{@}{\fw optional}}}

\subsubsection{{{ \fw documentclass}}}

{\color{orange}{{@}{\fw optional}}}

{\color{green}{{@}{\fw standard}}}

\subsubsection{{{ \fw img}}}

{\color{orange}{{@}{\fw optional}}}

\subsection{{{ \fw dst}}}

{\color{red}{{@}{\fw !optional}}}

\section{Next}

\section{The implementation}

\subsection{The parser}

\subsubsection{Case HTML comments}

\subsection{The writer}

\subsection{Utilities}

\subsubsection{HTML Comments}

\noindent\rule{\textwidth}{0.4pt}\end{document}
